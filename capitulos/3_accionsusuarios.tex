\chapter{Accións e usuarios}
A acción do xogo desenvólvese mediante accións, que é o que comunica aos
usuarios co mundo. Os usuarios escriben o que queren facer cos seus personaxes,
e é labor do DX executar o código pertinente e mostrarlles unha narración
por escrito que describa o sucedido.
\section{Decisións}
Un xogador nun momento dado pode escribir o que desexa que o seu personaxe faga.
Isto no Demiurgo coñécese como {\it decisión}. Este texto non ten ningún tipo de
efecto no sistema, simplemente é mostrado ao DX cando abre o escenario no que se
atopa o obxecto do xogador.

\section{Accións}
Por acción enténdese todo o sucedido entre dúas narracións, é dicir, todo o
código executado e as decisións correspondentes que o provocaron. O DX pode
executar todo o código que precise, e facer varias execucións diferenciadas,
antes de engadir unha narración e concluír por tanto a acción.
\subsection{Testemuñas}
As testemuñas son os xogadores que se atopan no escenario no momento de rematar
a acción. Estas testemuñas recóllense xusto antes de executar o último código,
o que iniciará o proceso de narración; polo que se incluirán entre elas os
personaxes cuxos obxectos cambiaran de escenario no último momento.

\section{Narracións}
A narración é o resultado textual da acción. O DX redáctao baseándose no estado
do escenario e nos {\it echos} (ver sección \ref{sec:echo}) producidos. Os
xogadores que sexan testemuñas poderán ver este resultado textual.
\subsection{Fragmentos ocultos}
O DX poderá marcar fragmentos da narración como ocultos empregando as etiquetas
[o][/o]. Por exemplo:
\begin{lstlisting}
Este texto e visible para todos.
[o=gamer]Este texto so e visible para o xogador 'gamer'.[/o]
[o=gamer user]
Este texto e visible para 'gamer' e 'user'.
[/o]
\end{lstlisting}