\chapter{Introdución}
A linguaxe COE (acrónimo de Código de Obxectos e Escenarios) é a linguaxe
empregada nas partidas no Demiurgo, o software de xestión de
partidas de rol asíncronas. O director de xogo ({\it DX}) pode escribir código
en COE para definir clases no mundo de xogo, e para realizar accións nos escenarios nas que
os distintos obxectos interactúan entre si.
\par
A finalidade deste manual é definir esta linguaxe e ensinar a darlle uso para
poder empregar todas as funcionalidades do Demiurgo sen dificultade, ofrecendo
para isto definicións formais, explicacións e exemplos en cada apartado. Grazas
a este sistema, o DX pode manter un mundo loxicamente coherente que lle faga de
soporte á hora de levar as partidas, evitando así erros e automatizando en gran
medida o apartado técnico das mesmas.