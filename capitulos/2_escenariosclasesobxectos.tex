\chapter{Escenarios, clases e obxectos}
Un mundo de xogo de Demiurgo está composto por obxectos, cada un dos cales
segue a definición dunha clase e atópase dentro dun escenario. A medida que
a partida avanza, os distintos obxectos créanse, modifícanse, móvense dun
escenario a outro e interactúan entre eles.
\section{Exemplo dunha situación}
Expresaremos unha situación normal do xogo mediante este exemplo:
\par
{\it A situación transcorre na taberna da capital do reino. Tres ananos están
nun recuncho discutindo entre eles acaloradamente. Detrás da barra pódese ver ao
taberneiro, un home corpulento que se distrae limpando un vaso. Detrás da barra
ten un mosquetón agochado por se as cousas se poñen feas, e por se acaso non
lle quita ollo aos ananos.}
\par
En base a esta descrición da situación, o director de xogo pode definir
internamente o seguinte:
\begin{itemize}
  \item Un escenario que se corresponde coa {\bf taberna}. O escenario debería
  ter un nome recoñecible para o director de xogo, como por exemplo {\it
  /reino/capital/taberna}, máis iso explicarase máis en detalle
  en \ref{subsec:nome-escenarios}.
  \item Como mínimo {\bf 5 obxectos}: \begin{itemize}
    \item Tres obxectos que representen aos {\bf ananos}. Xa que son tres
    obxectos moi semellantes, o normal é que pertenzan á mesma clase; por
    exemplo, unha clase chamada {\it Anano}. Máis isto non é unha norma,
    dependerá de como o director de xogo teña definido o mundo.
    \item Un obxecto que represente ao {\bf taberneiro}. Pode pertencer a unha
    clase diferenciada da dos ananos (como {\it Humano}); non obstante, ben poden
    pertencer tanto o taberneiro como os ananos a unha mesma clase {\it
    Intelixente} ou {\it Criatura}, por exemplo.
    \item Un obxecto que faga referencia á {\bf barra}, xa que o {\bf mosquetón}
    que contén é pode resultar relevante no futuro. Por tanto, o normal sería  que
    houbese un obxecto que representase á barra (de clase {\it Mobiliario} por
    exemplo), e no seu interior se atopase un inventario que conteña ao obxecto
    mosquetón (de clase {\it Mosquetón}). Explicarase con detalle o concepto dos
    inventarios en \ref{subsubsec:inventarios}.
  \end{itemize}
\end{itemize}
\par
Esta configuración é un simple exemplo que non ten por que cumprirse ao
100\% nunha partida real, pero da unha idea do concepto. Poñamos agora outro
exemplo: na mesma situación, comezan a suceder cousas.
\par
{\it Parece que a discusión dos ananos chega a un punto crítico. Dous deles
deciden simultaneamente que non poderán resolver as súas diferencias mediante a
diplomacia, polo que deciden sacar os puños e golpearse mutuamente. O
taberneiro, que non lles quitaba ollo, saca o mosquetón como resposta por se
comeza unha escalada de violencia.}
\par
Isto pode suceder por dous motivos: os dous ananos son personaxes
controlados por xogadores e ambos tomaron a mesma decisión (golpear ao outro) ao
mesmo tempo; ou ben son personaxes non xogadores ({\it NPC}, {\it non player
character}) e foi o propio DX o que tomou a decisión por eles (para que outros
xogadores no escenario presenciaran isto, por exemplo).
\par
En calquera caso, o que lle corresponde ao DX é executar o código necesario para
cambiar o estado do mundo. O código a redactar debe facer tres cousas:
\begin{enumerate}
  \item Un dos ananos debe {\bf golpear} ao outro, cos cambios que iso implique
  (por exemplo, reducir o seu campo {\it vida}).
  \item O segundo anano debe {\bf golpear} ao primeiro.
  \item O mosquetón debe deixar de estar no inventario da barra e debe
  {\bf moverse} ao inventario do taberneiro.
\end{enumerate}
\par
Este sería un exemplo de como podería ser este código:
\begin{lstlisting}
anano1.golpear(anano2)
anano2.golpear(anano1)
barra.contidos[0] >> taberneiro.inventario
\end{lstlisting}
\par
Como entendemos este código? Temos unha variable anano1 que fai referencia ao
obxecto {\it Anano} que da o primeiro golpe, unha variable anano2 que fai
referencia ao segundo, unha variable {\it barra}, unha variable {\it
taberneiro}, e un método {\it golpear()} definido para a clase {\it Anano}. Este
código enténdese así:
\begin{enumerate}
  \item O obxecto correspondente á variable anano1 executa o {\bf método}
  {\it golpear()}, collendo o obxecto da variable anano2 como argumento. Este
  método, definido na clase {\it Anano}, reduce a vida do anano seleccionado
  como argumento. Explicarase máis en detalle isto na sección
  \ref{subsec:metodos}.
  \item O mesmo que no anterior, pero ao revés.
  \item O primeiro obxecto ({\it [0]}) dos contidos da barra (''contidos'' é un
  inventario da clase {\it Mobiliario}), que neste caso entendemos que é o
  mosquetón, {\bf móvese} (>>) ao inventario do taberneiro. 
\end{enumerate}

\section{Escenarios}
\subsection{Definición de escenario}
O mundo de xogo está dividido en escenarios (tamén chamados {\it habitacións}
polo seu homólogo {\it rooms} en inglés). Cada escenario está composto por:
\begin{itemize}
  \item O seu nome, sobre o cal falaremos en \ref{subsec:nome-escenarios}.
  \item O conxunto de obxectos que se atopan ''dentro'' do escenario.
  \item As variables do escenario, as cales poden conter datos tales como
  números enteiros ou texto, pero tamén referencias a obxectos ou outros
  escenarios.
\end{itemize}
\par
O escenario forma unha unidade lóxica fundamental no mundo de xogo, e o habitual
é que as accións en cada escenario afecten só aos obxectos do propio escenario
(aínda que pode haber excepcións).

\subsection{Nome do escenario}
\label{subsec:nome-escenarios}
O nome de escenario serve para que o DX o identifique rapidamente. Ademais de
ofrecer un nome descritivo, o nome completo permite organizar os escenarios
nunha estrutura xerárquica, cunha lóxica semellante á das rutas de ficheiros nos
sistemas operativos.
\par
Exemplo:
\begin{lstlisting}
/continente/reinohumano/cidadecapital/armeria
/continente/reinohumano/cidadecapital/taberna
/continente/reinohumano/aldeacosteira
\end{lstlisting}
\par
Estes tres escenarios atópanse dentro de /continente/reinohumano, é dicir,
na estrutura lóxica do mundo de xogo están dentro do mesmo reino, o cal forma
parte do continente. A maiores, os dous primeiros están na mesma cidade. Estes
compoñentes normalmente reciben o nome de {\it rexións}, pero poden ter o
significado semántico que o DX de xogo prefira, por exemplo, nunha partida
futurista poderíamos atopar algo como:
\begin{lstlisting}
/planetamarte/coloniahumana/suburbios/casa3/cocinha
/planetamarte/orbita/sateliteflotante/laboratorio
/espazoexterior/zonadeasteroides
\end{lstlisting}
\par
Por último, o DX pode renunciar completamente a usar a xerarquía de escenarios,
aínda que iso pode dificultar o nomeado de escenarios:
\begin{lstlisting}
/chairas_da_capital_humana
/sala_do_gran_mago
\end{lstlisting}
\par
É importante ter en conta que o nome dos escenarios só pode estar composto por
caracteres alfanuméricos e barras baixas, e non pode conter letras que non
formen parte do alfabeto inglés (tales como {\it ñ} ou {\it ç}). O nome completo
sempre debe comezar por /. Por último, o Demiurgo non distingue entre
minúsculas e maiúsulas.
\par
Nota: {\it /} en si mesmo pode ser un escenario, o coñecido como {\it escenario
raíz}, pero non é habitual o seu uso.

\subsection{Contido do escenario}
O escenario pode conter obxectos (e de feito é habitual que os conteña). Estes
obxectos á súa vez poden dispoñer de inventarios (ver \ref{subsubsec:inventarios})
que conteñan outros obxectos. Mediante a interface gráfica, o DX pode ver os
distintos obxectos do escenario para operar con eles no código.

\subsection{Variables}
O DX pode definir variables no escenario a través do código. Estas variables
están accesibles dende o propio escenario e poden ter todo tipo de utilidades.
Por exemplo, é habitual referenciar os obxectos do escenario en variables para
poder manexalos co código máis facilidade.
\par
Exemplo: Imaxinemos que no escenario temos un obxecto que representa a un
trasno. O obxecto ten un identificador propio ({\it ID}) que é \#1234 (ver
\ref{sec:obxectos}).
No momento de querer actuar, se non empregamos variables precisamos referirnos a
el co seu ID:
\begin{lstlisting}
#1234.facerTrasnadas()
if(#1234.famento == true)
  ! "O trasno ten bastante fame"
\end{lstlisting}
\par
En cambio, se temos unha variable cun nome descritivo (por exemplo, {\it
trasno}) podemos codificalo así:
\begin{lstlisting}
trasno.facerTrasnadas()
if(trasno.famento == true)
  ! "O trasno ten bastante fame"
\end{lstlisting}
\par
Outro uso frecuente das variables é gardar datos. Neste exemplo tírase un dado
de 20 caras e móstrase un texto ou outro segundo o resultado:
\begin{lstlisting}
int resultado = d20
if(resultado > 10) {
  ! "A accion tivo exito"
}
if(resultado < 3) {
  ! "A accion foi un fracaso absoluto"
}
\end{lstlisting}



\section{Clases}
\label{sec:clases}
As clases son definicións empregadas a modo de modelo para crear obxectos. Todos
os obxectos que teñan a mesma clase terán unha serie de métodos e campos
comúns definidos na propia clase.
\par
Nesta sección explicarase como funcionan as clases e todo o relacionado con
elas. No capítulo \ref{ch:definicion} explícase polo miúdo a sintaxe formal para
crear clases.
\subsection{campos}
Os campos son variables vinculadas aos obxectos, e a súa finalidade é
almacenar datos relacionados con eles. Por exemplo, nun mundo de rol de fantasía
medieval, é habitual atopar nos personaxes campos como forza, axilidade ou
intelixencia. Nun mundo con máquinas de guerra, normalmente atoparemos campos
como a potencia de disparo ou o blindaxe. E tamén podemos atopar campos en
obxectos inanimados, como o peso, o prezo, a cor, etcéra.
\par
Os campos defínense coa clase, e cando é creado un obxecto desa clase,
automaticamente pasa a posuír todos os campos da mesma.
\par
Exemplo:
\begin{lstlisting}
elfo {
  int forza = 1
  int axilidade = 3
  int constitucion = 1
  int intelixencia = 2
  int vida
}
\end{lstlisting}
\par No exemplo vemos a definición dunha clase chamada {\it Elfo}. Como vemos,
os elfos teñen os campos forza, axilidade, constitución, intelixencia e vida,
os cales son números enteiros {\it int} e teñen asignados uns valores por
defecto (excepto o último).
Cando se crean obxectos da clase {\it Elfo} automaticamente xorden con eses
campos, pero poden ser modificados posteriormente.

\subsection{Métodos}
\label{subsec:metodos}
Cada clase pode ter definidos unha serie de métodos. Os métodos son fragmentos
de código relacionados coa clase que poden ser invocados en calquera momento do
xogo, útiles para simplificar procesos repetitivos.
\par Por exemplo:
\begin{lstlisting}
elfo {
  bendicir(elfo outro) {
    outro.vida = outro.vida + 5
    ! "O elfo foi bendecido"
  }
}
\end{lstlisting}
\par No exemplo, a clase {\it Elfo} dispón dun método {\it bendicir()} que recibe
como argumento un obxecto {\it Elfo} (outro elfo ou el mesmo!) e aumenta a súa
vida en 5. Deste modo, o DX non precisa escribir ese código cada vez que queira
facer , e pode invocalo do seguinte xeito:
\begin{lstlisting}
elfo1.bendicir(elfo2)
\end{lstlisting}
\subsubsection{Argumentos}
Vimos no exemplo do método {\it bendicir()} que o código fai referencia a unha
variable chamada {\it outro}. Isto é porque o ámbito dun método está reducido ao
obxecto que o invoca e aos argumentos que recibe; isto é, non pode acceder ás
variables do escenario.
\par
Cada método pode ter definidos un número indeterminado de argumentos de entrada,
e un argumento de saída (ou ningún). Todos estes argumentos deben especificar o
tipo, sexa un tipo primitivo como número enteiro ou texto, ou a clase se se
trata dun obxecto. Se a chamada do método se realiza con argumentos incorrectos
(por exemplo, enviando un obxecto dunha clase incorrecta, ou enviando un texto
cando require un número) o Demiurgo devolverá un erro.
\par
Dixemos que no ámbito dos métodos tamén está o obxecto que o invoca. Isto
faise mediante a variable {\it this}:
\begin{lstlisting}
elfo {
  ceder_vida(elfo outro) {
    outro.vida = outro.vida + 5
    this.vida = this.vida - 5
  }
}
\end{lstlisting}
\par
Os argumentos sempre teñen prioridade sobre os campos do obxecto en caso de que
os nomes coincidan; debido a isto, o uso de {\it this} é necesario para
referirse aos campos do obxecto cando isto suceda.

\subsubsection{Construtor}
\label{subsubsec:construtor}
Por último, hai un método especial moi importante chamado {\bf construtor}, que
se executa automaticamente no momento de crear o obxecto. É útil por tanto para
inicializar clases. O construtor recoñécese polo feito de chamarse igual ca
clase. No caso de ter argumentos, é necesario especificalos na creación do
obxecto:
\begin{lstlisting}
elfo {
  int idade
  str nome
  elfo(int novaidade, str novonome) {
    this.idade = novaidade
    this.nome = novonome
  }
}
\end{lstlisting}
\par E no momento de crealo:
\begin{lstlisting}
elfo legolas = new elfo(100, "Legolas o elfo")
\end{lstlisting}

\subsection{Herdanza de clases}
Un aspecto fundamental das clases é a herdanza das mesmas. Unha clase pode
especificarse como herdeira ou filla doutra clase, e deste modo herdará todos os
campos e métodos da clase pai. Esta característica, propia das linguaxes
orientadas a obxectos, permite aforrar moito traballo:
\begin{lstlisting}
animal {
  int altura
  str cor
  int peso
  str ruido
  
  facerRuido() {
  ! "este animal fai " + ruido
  }
}

can : animal {
  str tipoPelo
  str raza
  
  can() {
    ruido = "guau"
  }
}
\end{lstlisting}
\par
Outra utilidade da herdanza é que podemos gardar un obxecto da clase filla nunha
variable da clase pai. Por exemplo:
\begin{lstlisting}
personaxe {
  int vida = 10
}

guerreiro : personaxe {
  int ataque = 5
  int vida = 20
  atacar(personaxe outro) {
    outro.vida = outro.vida - ataque 
  }
}
\end{lstlisting}
\par Neste exemplo, o método {\it atacar()} pode recibir un personaxe como
argumento, ou calquera obxecto dunha clase filla (como outro guerreiro).

\subsubsection{Clase Object}
Realmente todas as clases herdan dalgunha outra clase. Se a clase pai non é
definida, esta será automaticamente a clase {\it Object}, unha clase auxiliar
que non conta con campos. É posible crear obxectos da clase Object
directamente, pero realmente non aportan demasiado ao xogo xa que non conteñen
valor semántico ningún.

\section{Obxectos}
\label{sec:obxectos}
Os obxectos son o elemento central do xogo. Cada obxecto representa a unha
entidade no xogo, que pode ser física (como un cervo ou unha mesa) ou un
concepto (como un estado alterado ou unha idea). Todos os obxectos teñen as
seguintes características:
\begin{itemize}
  \item Correspóndense cunha clase determinada.
  \item Atópanse nunha localización determinada, que pode ser un escenario, ou o
  inventario de outro obxecto.
\end{itemize}
\par
Os obxectos dispoñen dos campos definidos pola súa clase, e poden chamar
métodos da mesma. Estes aspectos explicáronse na sección ''Clases'' na sección
\ref{sec:clases}.

\subsection{Inventarios}
\label{subsec:inventarios}
Os obxectos poden dispoñer duns campos especiais que non existen no caso das
variables de escenario: os inventarios. Un inventario é unha localización
vinculada a un determinado obxecto. Os inventarios, como o resto de campos,
defínense na definición da clase, pero a diferencia do resto de campos non
poden ser reasignados: o obxecto nace cos seus
inventarios e só desaparecerán cando desapareza o obxecto.
\par Exemplo:
\begin{lstlisting}
heroe {
  % inventario
  % estados
  
  personaxe() {
    new espada() >> inventario
  }
}
\end{lstlisting}
\par
No exemplo podemos ver como a clase {\it Heroe} conta con dous
inventarios: un deles chámase ''inventario'' directamente, e o outro
''estados''. No construtor da clase (ver \ref{subsubsec:construtor})
especifícase que no momento de crearse o obxecto, crearase tamén un obxecto de
clase {\it Espada} e gardarase no inventario.
\par
Os inventarios, a pesar de ter este nome, poden ter o valor semántico que desexe
o DX: poden conter os tesouros que atopara o personaxe, conter unha lista de
estados alterados, conter as ideas dun NPC, etcétera. 