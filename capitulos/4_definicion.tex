\chapter{Definición da linguaxe COE}
\label{ch:definicion}
No capítulo anterior abordouse o funcionamento dos escenarios, as clases e os
obxectos. Neste capítulo mostrarase unha definición formal da linguaxe, tanto na
definición de clases como no código das accións.
\par
Nas descricións sintácticas é preciso ter en conta as seguintes marcas:
\begin{itemize}
  \item {\bf \textless elemento \textgreater}: Os elementos marcados deste
  xeito son elementos obrigatorios, pero non deben ser entendidos como código
  literal, xa que o contido é unha descrición do elemento. Exemplo: \textless
  argumento \textgreater.
  \item {\bf [ elemento ]}: Os elementos marcados deste xeito son opcionais.
  \item {\bf elemento \ldots}: Os elementos marcados deste xeito pódense repetir
  varias veces. Exemplo: \textless arg\textgreater {}[\textless arg\textgreater]
  \ldots.
\end{itemize}

\section{Tipos de datos}
No Demiurgo e na linguaxe COE podemos atopar os seguintes tipos de datos:
\begin{itemize}
  \item {\bf Número enteiro:} Representa un número sen decimais. Útil para facer
  cálculos, e para realizar tiradas de dados. Exemplo: {\it 27}. Tamén
  representa valores lóxicos, sendo 0 o equivalente a dicir ``falso'' e calquera
  outro valor (normalmente 1) o equivalente a dicir ``verdadeiro''. O seu tipo
  denótase coa palabra reservada {\it int}.
  \item {\bf Número con decimais:} Tamén coñecido como número con punto
  flotante ou simplemente flotante, representa un número non enteiro. Pode
  empregarse cando se empreguen campos que precisen decimais. Exemplo: {\it
  3.14}. Debe terse en conta que ao converter números con decimais a números
  enteiros, perderanse os decimais (non se redondea). O seu tipo
  denótase coa palabra reservada {\it float}.
  \item {\bf Cadea de texto:} Calquera fragmento de texto. Útil para mostrar
  mensaxes aos xogadores, ou para gardar anotacións do DX. Exemplo: {\it ''este
  é un texto de exemplo''}. O seu tipo
  denótase coa palabra reservada {\it str}.
  \item {\bf Escenario:} As variables e campos de escenario non conteñen o
  escenario como tal, senón unha referencia ao mesmo. Pode ser útil almacenar
  referencias a escenarios concretos para automatizar o movemento de obxectos
  entre eles. Aparte diso, non serven para facer operacións. Para obter
  unha referencia a un escenario concreto, emprégase o carácter {\it @}.
  Exemplo: {\it @/rexion1/cidade1/taberna}. O seu tipo
  denótase co carácter {\it @}.
  \item {\bf Inventario:} Exclusivo dos campos de obxecto, fan referencia ao
  inventario pertinente. Funcionan de maneira moi similar aos escenarios, salvo
  polo feito de que non se poden enviar como argumento nin sobreescribir o seu
  valor con outra referencia. O seu tipo
  denótase co carácter \%.
  \item {\bf Obxecto:} Referencia a un obxecto concreto. Permite interactuar co
  obxecto dende outro escenario, ou gardalo nunha variable cun nome descritivo.
  É posible obter unha referencia a un obxecto coñecendo a súa ID. Exemplo:
  \#{\it 1024}. O seu tipo
  denótase co nome da clase correspondente.
  \item {\bf Lista:} Lista de datos de calquera dos outros tipos. Exemplo:
  \{{\it 1, 2, 3, 4}\}. Tamén se poden crear listas baleiras: \{\}. O seu tipo
  denótase co tipo de elementos que contén e uns corchetes, por exemplo int[].
  Tamén se poden especificar listas de listas: int[][].
\end{itemize}


\section{Echo}
\label{sec:echo}
{\it Echo} é o termo que recibe o feito de emitir un resultado textual. No
Demiurgo, pódese facer echo de calquera valor que poida ser lido como unha cadea
de texto (tamén números enteiros e flotantes, e listas). Facer echo no código
ten unha utilidade: cando o DX estea redactando a narración da acción, poderá
ver todos os echos, polo que é útil para ver información antes de facer a
narración completa.
\par O echo faise co carácter reservado {\bf !}.
\begin{lstlisting}
! <valor_a_mostrar>
\end{lstlisting}

\section{Operacións}
No Demiurgo atopamos as seguintes operacións:
\begin{itemize}
  \item + {\bf Sumar:} Suma dous valores. Se os dous elementos son enteiros, o
  resultado é un enteiro; se un é flotante e o outro enteiro ou flotante, o
  resultado é un flotante. Se un dos elementos é unha cadea de texto, a
  operación no canto de ser unha suma de números convírtese nunha concatenación
  de texto: ``proba de '' + ``texto'' devolvería ``proba de texto'', e ``valor
  de '' + 4 devolvería ``valor de 4''. En calquera outro caso, emite un erro.
  \item - {\bf Restar:} Resta dous valores. Semellante a suma, salvo polo feito
  de que non serve para concatenar texto. Pódese usar cun só operando para obter
  o seu negativo.
  \item ! {\bf Operador lóxico ``non'':} Operador unario, é dicir, que acepta un
  operando. Devolve 1 se o dato é falso, ou 0 se o dato é verdadeiro. Un número
  enteiro ou flotante é verdadeiro se é distinto de 0. Unha lista devolverá unha
  lista de 1s  0s; o resto de tipos devolverá sempre un valor falso.
  \item == {\bf Igual:} Compara dous datos. Se os datos son loxicamente iguais,
  devolve un 1. En caso contrario, devolve un 0. Dúas referencias son iguais se
  referencian ao mesmo obxecto ou escenario.
  \item != {\bf Non igual:} Devolve 0 se os datos son loxicamente iguais, 0 en
  caso contrario.
  \item \textgreater {\bf Maior que:} Devolve 1 se o primeiro valor é maior co
  segundo, 0 en caso contrario. Funciona con enteiros e flotantes.
  \item \textless {\bf Menor que:} Devolve 1 se o primeiro valor é menor co
  segundo, 0 en caso contrario.
  \item \textgreater= {\bf Maior ou igual que:} Devolve 1 se o primeiro valor é maior ou
  igual co segundo, 0 en caso contrario.
  \item \textless= {\bf Menor ou igual que:} Devolve 1 se o primeiro valor é
  menor ou igual co segundo, 0 en caso contrario.
  \item \& {\bf Operador lóxico ``e'':} Compara dous elementos lóxicos. Se os
  dous son verdadeiros, devolve 1, en caso contrario, devolve 0.
  \item \textbar {\bf Operador lóxico ``ou'':} Compara dous elementos lóxicos.
  Se polo menos un dos dous é verdadeiro, devolve 1, en caso contrario, devolve
  0.
  \item = {\bf Asignar:} Asigna o valor da dereita á variable ou campo da
  esquerda. Ademais, se se usa en conxunto con outras operacións, devolve o
  valor asignado. En caso de non ser compatibles o valor e a variable, emite un
  erro.
  \item \textgreater\textgreater {\bf Mover:} Move o obxecto da esquerda á
  localización da dereita.
  \item ++ {\bf Concatenar:} Concatena dúas listas. Deben ser do mesmo tipo. Se
  un dos dous operandos non é unha lista pero ten o mesmo tipo que o tipo
  interno da lista, engádese ao comezo ou ao final (dependendo da orde dos
  operandos).
  \item D {\bf Tirada de dados:} Pódese usar como operador unario ou binario. Se
  se usa cun só operando, fai unha tirada dun dado onde o número de caras é o
  número especificado polo operando (que debe ser un enteiro) e devolve o
  resultado. Se se usa con dous operandos, fai X tiradas de Y caras, onde X é o
  valor do primeiro operando e Y o do segundo, e devolve unha lista de
  resultados.
\end{itemize}
\par A maiores, é posible recoller operacións entre parénteses para darlles
prioridade.
\par A prioridade de operadores é a seguinte:
\par
Tiradas de dados \textgreater parénteses \textgreater multiplicar ou dividir
\textgreater sumar ou restar \textgreater comparacións \textgreater operacións
lóxicas \textgreater asignacións \textgreater movementos
\par
Débese ter en conta que as asignacións sempre collerán unha variable do
lado esquerdo, polo que teñen a máxima prioridade por este lado.
\par Exemplos de operacións:
\begin{lstlisting}
(1+2*(3-4))                // Devolve -1
3d20 > 10                  // Devolve unha lista de 3 valores 0 ou 1
#42 >> @/zona1/estancia3   // Move o obxecto de ID 42 ao escenario
\end{lstlisting}
\subsection{Operacións con listas}
As listas permiten realizar practicamente todas as operacións dos elementos que
conteñen, pero teñen unha característica: o resultado cambia segundo a operación
sexa con outra lista ou cun elemento dun tipo distinto.
\subsubsection{Operacións entre elementos e listas}
Se se realiza unha operación entre un dato que non sexa lista e unha lista, o
Demiurgo o que devolverá será unha lista de resultados entre o dato e cada
elemento da lista.
\par Exemplo:
\begin{lstlisting}
3*{1,2,3}
\end{lstlisting}
\par Este exemplo devolvería a lista \{3,6,9\}.
\par Isto funciona tamén con listas de listas:
\begin{lstlisting}
5+{{5,5},{10,10}}
\end{lstlisting}
\par Este exemplo devolve a lista \{\{10,10\},\{15,15\}\}.
\subsubsection{Operacións entre listas}
As operacións entre listas simplemente devolven unha lista cos resultados
elemento a elemento. Se o tamaño das listas non coincide, devolverá un erro.
\par Exemplo:
\begin{lstlisting}
{2,3,6}+{1,2,5}
\end{lstlisting}
\par Este exemplo devolve a lista \{3,5,11\}.
\par Tamén valen combinacións entre os dous tipos de operación:
\begin{lstlisting}
{2,4}*{{10,10},{100,100}}
\end{lstlisting}
\par Devolvería a lista \{\{20,20\},\{400,400\}\}.

\section{Variables}
As variables de escenario créanse especificando o seu tipo e o nome da variable.
No momento de definir unha variable nova, é posible asignarlle un valor na mesma
liña:
\begin{lstlisting}
<tipo> <nome_variable> [ = <valor> ]
\end{lstlisting}
\par Exemplo de variables:
\begin{lstlisting}
int a               // define o enteiro 'a'
str b = "proba"     // define a cadea de texto 'b' co valor "proba"
elfo c = new elfo() // define a referencia 'c' e garda un novo Elfo 
\end{lstlisting}

\section{Condicionais}
É posible marcar fragmentos do código para que sexan executados ou non en base a
unha condición. Se a condición é verdadeira, o código execútase; en caso
contrario, non se executa nada. Tamén é posible especificar un código
alternativo para este último caso.
\par Definición formal:
\begin{lstlisting}
if (<condicion>) {
  <codigo_condicion_verdadeira>
}
[else {
  <codigo_condicion_falsa>
}]
\end{lstlisting}
\par Se o código só é unha liña, pódese prescindir das chaves.
\par Exemplo de condicional:
\begin{lstlisting}
if(d6 > 3) {  // tira un dado de 6 e o compara co enteiro 3
  ! "exito"
}
else {
  "fracaso"
}
\end{lstlisting}

\section{Bucles}
É posible especificar bucles no código, é dicir, fragmentos de código que se
executarán varias veces. O Demiurgo conta co bucle {\bf for}, que permite
executar un fragmento de código unha vez por cada elemento dentro dunha lista.
\par Definición formal:
\begin{lstlisting}
for (<variable> : <lista>) {
  <codigo_for>
}
\end{lstlisting}
\par Por cada elemento da lista, farase unha execución do código asignando tal
elemento á variable especificada. Se no canto dunha lista se lle pasa unha cadea
de texto, contará como unha lista de caracteres.
\par Se o código só é unha liña, pódese prescindir das chaves.
\par Exemplo de bucle:
\begin{lstlisting}
for(i : 5d10) {
  ! "resultado: " + i
}
\end{lstlisting}

\section{Funcións integradas}
É posible empregar funcións predefinidas no propio Demiurgo para realizar
determinadas operacións complexas. Estas funcións chámanse do seguinte xeito:
\begin{lstlisting}
[<saida>] = <nome_funcion> ( <arg> [, <arg> ] ... )
\end{lstlisting}
\par As funcións dispoñibles son as seguintes:
\begin{itemize}
  \item count({\it lista}): Recibe unha lista e devolve o número de elementos.
  Se o argumento é unha cadea de texto, devolve o número de caracteres.
  \item seq({\it inicio, fin}): Recibe dous números enteiros e devolve unha
  lista formada pola secuencia de enteiros entre ambos números, cun paso de 1. O
  número de inicio debe ser menor ou igual co de fin.
  \item reverse({\it lista}): Recibe unha lista e devolve a lista do revés, é
  dicir, co primeiro elemento en última posición e así sucesivamente. Se o
  argumento é unha lista de listas, as listas do interior non son modificadas,
  só a súa orde.
  \item sub({\it lista, inicio, fin}): Recibe unha lista e dous números
  enteiros e devolve unha sublista composta polos elementos entre a posición de inicio
  (inclusive) e a posición de fin (exclusive).
  \item sum({\it lista}): Recibe unha lista de enteiros e devolve a suma dos
  seus elementos.
  \item zeros({\it total}): Recibe un número enteiro e devolve unha lista
  composta por tantos ceros como especifique o argumento.
  \item destroy({\it obxecto}): Recibe unha referencia de obxecto, e destrúe ese
  obxecto. Desaparecerá do mundo de xogo, e con el os seus inventarios e os
  obxectos que estes conteñan. Esta función non devolve ningún valor.
\end{itemize}
\par Exemplo:
\begin{lstlisting}
sum( seq(1,10) )          // suma os enteiros do 1 ao 10
\end{lstlisting}

\section{Obxectos e métodos}
\subsection{Creación de obxectos}
Os obxectos créanse coa palabra reservada ``new'' o nome da clase, e se é
preciso os argumentos do método construtor. A definición é esta:
\begin{lstlisting}
new <nome_clase> ( [ <arg> [, <arg> ] ... ] )
\end{lstlisting} 
\par Ademais de crear o obxecto, este método devolve unha referencia ao mesmo,
que pode ser gardada nunha variable mediante unha asignación.
\par
Os obxectos no momento de ser creados aparecen no escenario no que se
estea escribindo o código.
\par Exemplo de creación de obxecto:
\begin{lstlisting}
elfo legolas = new elfo(100, "Legolas o elfo")
\end{lstlisting}

\subsection{Invocación de métodos}
Para invocar un método cun obxecto é preciso ter a referencia dese obxecto,
extraer o seu método co punto ({\bf .}) e engadir os argumentos en caso de ser
necesarios. Se o método ten argumento de saída, pódese gardar o seu valor nunha
variable, ou empregarse directamente para facer operacións.
\begin{lstlisting}
<referencia_obxecto>.<nome_metodo> ( [ <arg> [, <arg> ] ... ])
\end{lstlisting} 
\par Exemplo:
\begin{lstlisting}
legolas.bendicir(outroelfo)
\end{lstlisting}

\section{Definición de clases}
No momento de crear a clase, especifícanse os atributos e métodos que terán os
obxectos creados a partir dela. A definición dunha clase é a seguinte:
\begin{lstlisting}
<nome_clase> {
  <atributos>
  <metodos>
}
\end{lstlisting}
\par Os atributos defínense como as variables de escenario. Os métodos defínense
da seguinte forma:
\begin{lstlisting}
[<tipo> <arg_saida> =]
  <nome_metodo>([ <tipo> <arg>[, <tipo> <arg>]...])
{
  <codigo_metodo>
}
\end{lstlisting}
\par Se no código do método se asigna un valor ao argumento de saída, este
devolverase cando o método sexa chamado.
\par Exemplo de clase:
\begin{lstlisting}
anano : criatura {
  int forza = 3
  int axilidade = 1
  int intelixencia = 1
  int vida = 30
  str nome
  
  anano(str nome) {
    this.nome = nome
  }
  
  int dano = golpear(criatura outro) {
    dano = sum((forza)d6)
    criatura.vida = criatura.vida - dano
  }
}
\end{lstlisting}

\section{Usuarios}
Os usuarios non poden ser manexados como valores por ningún tipo de dato; pero
hai unha operación especial ({\bf -\textgreater}) que permite asignarlle un
obxecto a un usuario coñecendo o seu nome.
\begin{lstlisting}
$<nome_usuario> -> <referencia_obxecto>
\end{lstlisting}
\par Exemplo:
\begin{lstlisting}
$gamer -> #30
\end{lstlisting}